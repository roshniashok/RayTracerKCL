\documentclass{article}
\usepackage[utf8]{inputenc}
\usepackage{graphicx}
\usepackage{biblatex}


\title{Ray Tracer}
\author{MiddleEast Group }
\date{February 2018}

\begin{document}
\maketitle 



\section{Project Description}

\subsection{Introduction}
\paragraph{}
Ray Tracing is a rendering technique which takes simple 2D objects such as circles, squares or rectangles and
produces 3D rendering of these objects in the form of spheres, cubes, cuboids etc.  by tracing the path of light as pixels on a screen space and imitating the effects of the light  with 3D objects. Scenes in ray tracer are defined mathematically by the developer. In addition, it can take data from images and models captured by means such as digital photography. \newline An advantage of ray tracing is that it gives a more realistic and desirable method of rendering. Many physically correct experience can be demonstrated with ease using a ray tracing algorithm, since the algorithm simulates the sign of light in the real world.
However, it is extremely time consuming to find the intersection between rays and geometry. The main issue with ray tracing is the speed. Anyhow, as computers become more quicker, it is less and less of an issue. 

\subsection{What is Ray Tracer?}

\paragraph{}
To understand ray tracer you need to know what is "ray"? Basically, ray is a line that starts off at a point into a space towards some direction. When a ray hits a surface its either reflected or refracted.
In the real world, light is reflected on object surfaces until it noticed by the observer (camera or person). \newline Now, this method can not be used while using a computer to simulate a ray tracer. Hence, a revers mechanism is used where the rays are sent out from the observer viewpoint. This is much better in terms of computational speed. 
\paragraph{}
Ray tracer technique is based on an algorithm where its builds an image pixel by pixel. It casts additional rays from the hit point to determine the pixel color as shown in Figure 1.
Once the nearest object has been identified, the algorithm will estimate the incoming light at the point of intersection, examine the material properties of the object, and combine this information to calculate the final color of the pixel. In other words we can say that if the light hits the surface and not be blocked,  The shading of the surface is computed using traditional 3D computer graphics shading models. 


\begin{figure}
    \centering
    \includegraphics{F1.png}
    \caption{Basic ray tracing}
    \label{figure1}
    \soucre{
http://i.imgur.com/YPSKk.png}
\end{figure}

\subsection{Objective}

\paragraph{}
This project aims at developing a Ray Tracer software which consists of two parts. The first part is the GUI, that will allow the end users to insert preferable values in the form of coordinates (x, y) which will be sent to the back-end system as a text file.
The second part, which is the back-end system will take the input and produce  a 3D image for the end user and this image will be saved as PNG file. 
The software will be based on C++ language since all the team members are confident with this language and it is proved that it is fast compared to other programming languages. It has a compiling time of 1.5 seconds according to \textit{ffconsultancy.com }.

\subsection{Strategy}
\paragraph{}
Middle East group has agreed to meet once a week for two hours. Meetings will be held in college's study rooms where we will discuss and assign tasks among the team members. The tasks will be assigned based on the teammate's skills and the load and difficulty of the task. Moreover,  a decision will be taken only after all the team members agreed to it. Otherwise, the decision agreed by majority of the team members will be taken.


\subsection{Time Table}

\begin{center}
\begin{tabular}{ |c|c|c|} 
\hline
\textbf{Week no.} & \textbf{Week Date } & \textbf{To-do } \\ 
\hline
Week 1 & 14 January  & Brainstorming \\ 
\hline
Week 2 & 21 January  & Discussion  on tools and language to be used. \\ 
\hline

Week 3 & 28 January & Search and share knowledge. 
 Also, starting with GUI and Back-end system. 
 \\ 
\hline
Week 4 & 4 February & Finalizing the initial report and presentation.  \\ 
\hline
Week 5 &  11 February & Working on the software (GUI and Back-end)  \\ 
\hline
Week 6 & 18  February & Working on the software (GUI and Back-end)   \\ 
\hline
Week 7 & 25  February & Working on the software (GUI and Back-end)   \\ 
\hline
Week 8 & 4 March  & Working on the software (GUI and Back-end)   \\ 
\hline
Week 9 & 11 March  & Testing and handling errors.   \\ 
\hline
Week 10 & 18 March  & Preliminary final report. \\ 
\hline
Week 11 & 25 March  & *Finalizing the project  \\ 

\hline

\end{tabular}
\end{center}

*(Check the system is running successfully and practice for the presentation) 

\subsection{Progress}

\paragraph{}
Currently, team has searched and studied the basic concepts of ray tracers. Many of the open-source codes helped us to understand the logic behind the program and algorithm used. The team has started working on the GUI to provide an interactive experience to the user where he/she can enter the length, width and the color values. In addition, the user can select the preferred shape from a drop down menu. 
\newline 
The team is working on the GUI and the back-end software parallelly to ensure that back-end corresponds to the GUI and therefore does not give rise to problems in the future. We have considered taking an open source project for our project as it is a common algorithm and it's used in all rendering methods. We would like to make modifications to the aforementioned project to align with our GUI and making it more user-responsive, as well as adding new features to enhance the program. 




\section{Project Organization}

\subsection{Roles}
Team has decided to divide the technical work into two halves. Half of the team will work on GUI and the other half will handle the back-end system, as well as the report and the presentation.  
\begin{itemize}

\item \textbf{GUI and report}: Norah Alsuwily, Maryam Yasin, Sara Alotaibi. 
\item \textbf{Back-end and presentation}: Roshini Ashokkumar, Nandhini Sreekumar, Tamanna Qureshi. 

\end{itemize}

This way we are able to efficiently manage the load of this project to ensure that every team member is allocated with balanced workload. Thus, ensuring that the project proceed smoothly and successfully. 


\subsection{Processes and Tools}

This section tells about the team's operations and tools used to manage the work as below.   
\begin{itemize}

\item Meeting and Communication: Slack, Whatsapp and college's meeting rooms.
\item Development: QT Creator, Visual Studio and CodeBlocks.
\item Documentation: Latex and Powerpoint.

\end{itemize}


\subsection{Meeting Management} 
Team has agreed to meet regularly every week on Tuesday and divide the tasks fairly according to each teammate's skills. The progress will be monitored every week and comments from every team member will be adhered to. \newline If there are any personal circumstances that will prevent the teammates from completing the task, she will need to follow it up in the following week and failure to do so, will definitely reflect on her marks based on Berger algorithm at the end of the project as the marks are based on her work and efforts that she has put into this project in this way we can efficiently handle peer assessment without any conflicts. 



\end{document}

